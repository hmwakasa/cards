% +++
% latex="lualatex"
% +++
\documentclass[line_length=50zw,head_space=2cm,foot_space=1cm]{jlreq}

\usepackage{luatexja}
\ltjdefcharrange{11}{`→,`↑,`↓,`←}
\ltjsetparameter{jacharrange={-2,-8,+11}}
\usepackage[no-math,match,deluxe,jfm_yoko=jlreq]{luatexja-preset}
\usepackage{luatexja-otf,luatexja-adjust}
\newopentypefeature{PKana}{On}{pkna}

\usepackage[nowidering]{yhmath}
\usepackage{newpxmath,amsmath,mathtools,mleftright}
\usepackage[T1]{fontenc}
\usepackage[notrig,italicdiff]{physics}
\mleftright

\usepackage[loadonly,]{enumitem}
\newlist{desc}{description}{5}
\setlist[desc]{labelindent=1\zw,labelsep*=1\zw,labelwidth=4\zw}
\newlist{enu}{enumerate}{5}
\setlist[enu]{label*=\arabic*.}

\usepackage[scr=boondoxo,frak=pxtx,bb=mth]{mathalfa}
\DeclareMathAlphabet{\mathnormal}{T1}{pplx}{m}{it}
\DeclareMathAlphabet{\mathrm}{T1}{pplx}{m}{n}
\DeclareMathAlphabet{\mathit}{T1}{pplx}{m}{it}
\DeclareMathAlphabet{\mathtt}{T1}{lmtt}{m}{n}
\DeclareMathAlphabet{\mathsf}{T1}{kurier}{m}{n}
\DeclareMathAlphabet{\mathbsf}{T1}{kurier}{b}{n}
\DeclareMathAlphabet{\mathbold}{T1}{pplx}{b}{it}
\DeclareMathAlphabet{\mathbf}{T1}{pplx}{b}{n}
\DeclareMathAlphabet{\mathscr}{U}{BOONDOX-calo}{m}{n}
\DeclareMathAlphabet{\mathbscr}{U}{BOONDOX-calo}{b}{n}
%\DeclareMathAlphabet{\mathcal}{OT1}{eusm10}{m}{n}
%\DeclareMathAlphabet{\mathbcal}{OT1}{eusm10}{b}{n}
\DeclareMathAlphabet{\mathfrak}{OT1}{tx-frak}{m}{n}
\DeclareMathAlphabet{\mathbfrak}{OT1}{tx-frak}{b}{n}
\DeclareMathAlphabet{\mathbb}{U}{dsss}{m}{n}
\DeclareSymbolFont{operators}{T1}{uop}{m}{n}

\newfontfamily{\hmuopfont}{FOT-RodinNTLGPro}[NFSSFamily=hmrn]
%\DeclareSymbolFont{numbers}{T1}{hmrn}{m}{n}
\DeclareSymbolFont{numbers}{T1}{pplx}{m}{n}
\DeclareMathSymbol{0}\mathalpha{numbers}{`0}
\DeclareMathSymbol{1}\mathalpha{numbers}{`1}
\DeclareMathSymbol{2}\mathalpha{numbers}{`2}
\DeclareMathSymbol{3}\mathalpha{numbers}{`3}
\DeclareMathSymbol{4}\mathalpha{numbers}{`4}
\DeclareMathSymbol{5}\mathalpha{numbers}{`5}
\DeclareMathSymbol{6}\mathalpha{numbers}{`6}
\DeclareMathSymbol{7}\mathalpha{numbers}{`7}
\DeclareMathSymbol{8}\mathalpha{numbers}{`8}
\DeclareMathSymbol{9}\mathalpha{numbers}{`9}

%% フォントがない場合は以下の5行を削除
\setsansjfont[Ligatures=TeX,BoldFont=RodinNTLGPro-EB]{FOT-RodinNTLGPro-DB}
\setmainjfont[Ligatures=TeX,BoldFont=RodinNTLGPro-EB]{FOT-RodinNTLGPro-DB}
\setsansfont[Ligatures=TeX,BoldFont=RodinNTLGPro-EB]{FOT-RodinNTLGPro-DB}
\setmainfont[Ligatures=TeX,BoldFont=RodinNTLGPro-EB]{FOT-RodinNTLGPro-DB}
\ltjsetparameter{yjabaselineshift=0pt,yalbaselineshift=0.5pt}

\usepackage{scalefnt}
\usepackage{multirow}
\usepackage{multicol}
\ltjenableadjust[lineend=extended,priority=true,profile=true,linestep=true]
\allowdisplaybreaks[4]

\usepackage{hmtrump}

\newcommand{\hmG}{%
	\tikz[baseline=(T.base)]
	\node[fill=black,text=white,outer sep=0pt,inner sep=0.1ex]
	(T)at(0,0){G};%
}

%%%%%%%%%%%%%%%%%%%%%%%


\begin{document}

\pagestyle{empty}

\vspace{-1cm}
\title{Japanisch Vier--Anderle ルールサマリー}
\author{ひとみさん}

\begin{center}
{\LARGE Japanisch Vier--Anderle ルール早見}---ひとみさん \today
\end{center}

\setlength{\parindent}{0pt}

\section{用意するもの}
\textbullet チップ(一人あたり\((\text{人数}+\text{1})\times\text{4}\)枚くらい)\quad
\textbullet タロットカード(切札と絵札)\\
タロットカードの強さは:
\tikz[baseline=(T.base)]{\draw node[inner sep=0pt,outer sep=0pt](C){\large ■}node(T)[color=white]{強};}
\textbf{切札}\tarottrump{0} \tarottrump{21} \tarottrump{20} \tarottrump{19}
…… \tarottrump{2} \tarottrump{1}
\textbf{絵札}\trump{K}{x} \trump{Q}{x} \trump{C}{x} \trump{J}{x}
\tikz[baseline=(T.base)]{\draw node[inner sep=0pt,outer sep=0pt](C){\large □}node(T)[]{弱};}

\section{フォローの規則}
\begin{itemize}
	\item 切札を出すときは、そのトリックに現時点で勝てる切札を出す
	\item 切札リードの場合
		\begin{itemize}
			\item 切札を出す
			\item 切札がなければ好きに出す
		\end{itemize}
	\item 絵札リードの場合
		\begin{itemize}
			\item リードのスートのカードを出す
			\item リードのスートをがなければ切札を出す
			\item リードのスートも切札もなければ好きに出す
		\end{itemize}
\end{itemize}

\section{ディールの流れ}
\begin{center}
	\begin{tikzpicture}[x=6\zw,y=-4\zh]
		\draw(0,0)--(4,1)--(8,0)--cycle node[midway,below]{ポットにチップが};
		\draw(0,0)|-(4,1)node[midway,above right]{ない}-|(8,0)node[midway,above left]{ある};
		\draw(0,2)rectangle(4,1)node[midway]{ディーラーがポットに 4 チップ入れる};
		\draw(0,3)rectangle(4,2)node[midway]{カードを 4 枚配る};
		\draw(0,3)--(2,4)--(4,3)--cycle node[midway,below]{TT に参加するか※};
		\draw(0,3)|-(2,4)node[midway,above right]{降りる}-|(4,3)node[midway,above left]{出る};
		\draw(0,8)rectangle(2,4)node[midway,text width=12\zw,align=flush center]{サイドポットに\\4 チップ入れる};
		\draw(2,5)rectangle(4,4)node[midway]{トリックテイキング※};
		\draw(2,5)--(3,6)--(4,5)--cycle node[midway,below]{トリックを取ったか};
		\draw(2,5)|-(3,6)node[midway,above right]{No}-|(4,5)node[midway,above left]{Yes};
		\draw(2,8)rectangle(3,6)node[midway,text width=6\zw,align=flush center]{サイドポット\\に 8 チップ\\入れる};
		\draw(3,8)rectangle(4,6)node[midway,text width=6\zw,align=flush center]{1 トリック\\につき\\1/4 ポット\\得る};
		\draw(0,9)rectangle(4,8)node[midway]{サイドポットをポットに移す};
		\draw(4,2)rectangle(8,1)node[midway]{ディーラーがポットに 4 チップ入れる};
		\draw(4,3)rectangle(8,2)node[midway]{カードを 4 枚配る};
		\draw(4,5)rectangle(8,3)node[midway]{トリックテイキング※};
		\draw(4,5)--(6,6)--(8,5)--cycle node[midway,below]{トリックを取ったか};
		\draw(4,5)|-(6,6)node[midway,above right]{No}-|(8,5)node[midway,above left]{Yes};
		\draw(4,9)rectangle(6,6)node[midway,text width=12\zw,align=flush center]{ポットに\\4 チップ入れる};
		\draw(6,9)rectangle(8,6);
	\end{tikzpicture}
\end{center}
※トリックテイキングに参加するかは、ディーラーの隣のプレイヤーから順に判断\\
※オープニングリードは、参加している人でディーラーの次に一番近い人

\begin{multicols}{2}
	\columnbreak
\end{multicols}
\end{document}
