%! LuaLaTeX 文書
\documentclass[jafontsize=8pt]{jlreq}

\usepackage{luatexja}
\ltjdefcharrange{11}{`→,`↑,`↓,`←}
\ltjsetparameter{jacharrange={-2,-8,+11}}
\usepackage[no-math,match,deluxe,jfm_yoko=jlreq]{luatexja-preset}
\usepackage{luatexja-otf,luatexja-adjust}
\newopentypefeature{PKana}{On}{pkna}

\usepackage{yhmath,amssymb,mathtools,mathabx,mathrsfs,mathbbol}

\usepackage[math]{iwona}
\usepackage[euler-digits]{eulervm}
\usepackage{yhmath}
\usepackage[scaled]{beramono}
\DeclareMathAlphabet{\mathtt}{T1}{fvm}{m}{n}
\DeclareMathAlphabet{\mathsf}{T1}{uop}{m}{n}
%% フォントがない場合は以下の5行を削除
\setsansjfont[Ligatures=TeX,BoldFont=RodinNTLGPro-B]{FOT-RodinNTLGPro-DB}
\setmainjfont[Ligatures=TeX,BoldFont=RodinNTLGPro-B]{FOT-RodinNTLGPro-DB}
\setsansfont[Ligatures=TeX,BoldFont=RodinNTLGPro-B]{FOT-RodinNTLGPro-DB}
\setmainfont[Ligatures=TeX,BoldFont=RodinNTLGPro-B]{FOT-RodinNTLGPro-DB}
\ltjsetparameter{yjabaselineshift=0pt,yalbaselineshift=0.5pt}

\usepackage{scalefnt}
\usepackage{multirow}
\usepackage{multicol}
\ltjenableadjust[lineend=extended,priority=true,profile=true,linestep=true]
\allowdisplaybreaks[4]

\usepackage{hmtrump}

\usepackage[a4paper,margin=15mm]{geometry}

\newcommand{\hmG}{%
	\tikz[baseline=(T.base)]
	\node[fill=black,text=white,outer sep=0pt,inner sep=0.1ex]
	(T)at(0,0){G};%
}

%%%%%%%%%%%%%%%%%%%%%%%


\begin{document}

\pagestyle{empty}

\vspace{-1cm}
\title{Skat ルールサマリー}
\author{ひとみさん}

\begin{center}
{\LARGE Skat ルール早見}---ひとみさん \today
\end{center}

\setlength{\parindent}{0pt}

\section{ゲームの流れ}
\textbf{\mbox{カードを配る}\hfill →\hfill \mbox{ビッドをする}\hfill →\hfill 
\mbox{スカート交換、ゲームの種類を決める}\hfill →\hfill \mbox{トリックテイキング}\hfill
→\hfill \mbox{得点計算}\hfill}\\
配り方: 3→スカート2→4→3
\hspace*{1\zw}(4人以上)ディーラーと(5人)ディーラーの3つ左隣は不参加、ディーラーの次がオープニングリード\\
ビッドは勝ち抜き方式(先のビッドに優先権)

\section{カード}
基本の目的: 61カード点以上獲得/ヌルゲーム: 1トリックも取らない
\begin{table}[h]
\begin{minipage}{.2\textwidth}
\centering
\caption{カード点}
\begin{tabular}{c|c}
\hline
\trump Ax&11\\
\trump Tx&10\\
\trump Kx&4\\
\trump Qx&3\\
\trump Jx&2\\
\hline\hline
総計&120\\
\hline
\end{tabular}
\end{minipage}
\begin{minipage}{.75\textwidth}
\centering
\caption{ランクの順}
\begin{tabular}{c|c|l}
\hline
\multirow{2}{*}{スートゲーム}&
切札&$\trump JC>\trump JS>\trump JH>\trump JD
	>\trump Ax>\trump Tx>\trump Kx>\trump Qx>\trump 9x>\trump 8x>\trump 7x$\\
&切札以外&$\trump Ax>\trump Tx>\trump Kx>\trump Qx>\trump 9x>\trump 8x>\trump 7x$\\
\hline
\multirow{2}{*}{グランド}&
切札&$\trump JC>\trump JS>\trump JH>\trump JD$\\
&切札以外&$\trump Ax>\trump Tx>\trump Kx>\trump Qx>\trump 9x>\trump 8x>\trump 7x$\\
\hline\hline
\multicolumn{2}{c|}{ヌルゲーム}&
$\trump Ax>\trump Kx>\trump Qx>\trump Jx>\trump Tx>\trump 9x>\trump 8x>\trump 7x$\\
\hline
\end{tabular}
\end{minipage}
\end{table}

\section{ゲーム点}
\textbf{$\text{Gp}=\text{基本点}\times\text{達成点}$、失敗ならば2倍}\\
ビッドの点数に足りず失敗→達成点を増やし、ビッドの点数を超えたところで Gp とし、それの2倍の失点\\
\textbf{シュワルツ}→全トリック取る/\textbf{シュナイダー}→90点以上/\textbf{ウベア}→手札を公開して全トリック\\
\hspace{1\zw}いずれもハンドゲームのときにのみ宣言可能(宣言せずとも、成り行きで達成は可能)\\
\textbf{ウィズ/ウィズアウト}→上から何枚の連続した切札を持っている/いないか
\begin{table}[h]
\begin{minipage}{.2\textwidth}
\centering
\caption{基本点}
\begin{tabular}{cc|c}
\hline
\multirow{2}{*}{\mbox{\tate \hspace{0.3\zw}スートゲーム}}
&\hmD&9\\
&\hmH&10\\
&\hmS&11\\
&\hmC&12\\
\hline
\multicolumn{2}{c|}{グランド}&24\\
\hline
\end{tabular}
\end{minipage}
\begin{minipage}{.75\textwidth}
\centering
\caption{達成点}
\begin{tabular}{cc||cccccc}
\hline
&&シュワルツ&シュナイダー&89--31&逆シュナイダー&逆シュワルツ\\
\hline\hline
\multicolumn{2}{c||}{スカートゲーム}
&3&2&1&2&3\\
\multirow{4}{*}{\mbox{\tate \hspace{0.3\zw}ハンドゲーム}}
&宣言無し
&4&3&2&3&4\\
&シュナイダー宣言
&5&4&4&5&6\\
&シュワルツ宣言
&6&6&6&7&8\\
&ウベア宣言
&7&7&7&8&9\\
\hline
\multicolumn{7}{r}{+ウィズ/ウィズアウト}
\end{tabular}
\end{minipage}
\end{table}
\begin{table}[h]
\caption{基本点(ヌルゲーム)→達成点は1固定}
\centering
\begin{tabular}{cccc}
\hline
ヌル(ハンド)&ヌル(スカート)&ヌル・ウベア(ハンド)&ヌル・ウベア(スカート)\\
\hline
23&35&46&59\\
\hline
\end{tabular}
\end{table}

\section{ビッド可能な点数}
18 \hmD2→\hmD のスートゲーム、達成点2で 18 Gp\hspace{1\zw}\hmG →グランド\\
\begin{multicols}{4}
\renewcommand{~}{\hspace{1\zw}}
18~\hmD2\\
20~\hmH2\\
22~\hmS2\\
23~ヌル(スカート)\\
24~\hmC2\\
27~\hmD3\\
30~\hmH3\\
33~\hmS3\\
35~ヌル(ハンド)\\
36~\hmD4, \hmC3\\
40~\hmH4\\
44~\hmS4\\
45~\hmD5\\
46~ヌル・ウベア(ハンド)\\
48~\hmC4, \hmG2\\
50~\hmH5\\
54~\hmD6\\
55~\hmS5\\
59~ヌル・ウベア(ハンド)\\
60~\hmH6, \hmC5\\
63~\hmD7\\
66~\hmS6\\
70~\hmH7\\
72~\hmD8, \hmC6, \hmG3\\
\begin{center}$\vdots$\end{center}
\end{multicols}


\end{document}