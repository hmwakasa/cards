%! LuaLaTeX 文書
\documentclass[jafontsize=12pt]{jlreq}

\usepackage{luatexja}
\ltjdefcharrange{11}{`→,`↑,`↓,`←}
\ltjsetparameter{jacharrange={-2,-8,+11}}
\usepackage[no-math,match,deluxe,jfm_yoko=jlreq]{luatexja-preset}
\usepackage{luatexja-otf,luatexja-adjust}
\newopentypefeature{PKana}{On}{pkna}

\usepackage{yhmath,amssymb,mathtools,mathabx,mathrsfs,mathbbol}

\usepackage[math]{iwona}
\usepackage[euler-digits]{eulervm}
\usepackage{yhmath}
\usepackage[scaled]{beramono}
\DeclareMathAlphabet{\mathtt}{T1}{fvm}{m}{n}
\DeclareMathAlphabet{\mathsf}{T1}{uop}{m}{n}
%% フォントがない場合は以下の5行を削除
\setsansjfont[Ligatures=TeX,BoldFont=RodinNTLGPro-B]{FOT-RodinNTLGPro-DB}
\setmainjfont[Ligatures=TeX,BoldFont=RodinNTLGPro-B]{FOT-RodinNTLGPro-DB}
\setsansfont[Ligatures=TeX,BoldFont=RodinNTLGPro-B]{FOT-RodinNTLGPro-DB}
\setmainfont[Ligatures=TeX,BoldFont=RodinNTLGPro-B]{FOT-RodinNTLGPro-DB}
\ltjsetparameter{yjabaselineshift=0pt,yalbaselineshift=0.5pt}

\usepackage{scalefnt}
\usepackage{multirow}
\usepackage{multicol}
\ltjenableadjust[lineend=extended,priority=true,profile=true,linestep=true]
\allowdisplaybreaks[4]

\usepackage{hmtrump}

\usepackage[a4paper,margin=10mm]{geometry}

\newcommand{\cellalign}[2]{\multicolumn{1}{#1}{#2}}

%%%%%%%%%%%%%%%%%%%%%%%


\begin{document}

\pagestyle{empty}

\begin{center}
{\LARGE ミゼルカ ルール早見}\\ひとみさん \today
\end{center}

\setlength{\parindent}{0pt}
\section{得点表}
{\LARGE\rule{0pt}{0pt}\hfill 目標: \textbf{7---5---1}\hfill\rule{0pt}{0pt}}
\begin{table}[h]
	\centering\Large
	\begin{tabular}{|c||p{1.8cm}|p{1.8cm}|p{1.8cm}|p{1.8cm}|p{1.8cm}|p{1.8cm}|c|}
		\hline
		プレイヤー&{\hfil\hmS\hfil}&{\hfil\hmH\hfil}&{\hfil\hmD\hfil}&{\hfil\hmC\hfil}&{\hfil{\hmtcfont NT}\hfil}&{\hfil{\hmtcfont Miz.}\hfil}&合計点\\
		\hline\hline
		&□&□&□&□&□&□&\\\cline{2-7}
		&&&&&&&\\\cline{2-7}
		&&&&&&&\\\cline{2-7}
		\hline\hline
		&&&&&&&\\\cline{2-7}
		&□&□&□&□&□&□&\\\cline{2-7}
		&&&&&&&\\\cline{2-7}
		\hline\hline
		&&&&&&&\\\cline{2-7}
		&&&&&&&\\\cline{2-7}
		&□&□&□&□&□&□&\\\cline{2-7}
		\hline
	\end{tabular}
\end{table}

\section{ディールの始め方}
\begin{enumerate}
	\item 全員と中央に\textbf{6枚}ずつ配る
	\item デクレアラーがプレイの種類を決定
	\item 残りを配り切る
	\item カードを交換(デクレアラーから順番に、中央にあるカードの枚数まで)
\end{enumerate}

% \section{ゲームの種類}
% \begin{table}[h]
% 	\begin{tabular}{cl}
% 		\hmS\hmH\hmD\hmC&ぞれぞれのスートを切札にして、トリックを取る\\
% 		{\hmtcfont NT}&切札無しでトリックを取る\\
% 		{\hmtcfont Miz.}&切札無しでトリックを取らない\\
% 	\end{tabular}
% \end{table}

\end{document}
