% +++
% latex="lualatex"
% +++
\documentclass[line_length=50zw,head_space=2cm,foot_space=2cm]{jlreq}

\usepackage{luatexja}
\ltjdefcharrange{11}{`→,`↑,`↓,`←}
\ltjsetparameter{jacharrange={-2,-8,+11}}
\usepackage[no-math,match,deluxe,jfm_yoko=jlreq]{luatexja-preset}
\usepackage{luatexja-otf,luatexja-adjust}
\newopentypefeature{PKana}{On}{pkna}

\usepackage[nowidering]{yhmath}
\usepackage{newpxmath,amsmath,mathtools,mleftright}
\usepackage{tabularray}
\usepackage[notrig,italicdiff]{physics}
\mleftright

\usepackage[loadonly,]{enumitem}
\newlist{desc}{description}{5}
\setlist[desc]{labelindent=1\zw,labelsep*=1\zw,labelwidth=4\zw}
\newlist{enu}{enumerate}{5}
\setlist[enu]{label*=\arabic*.}

\usepackage[scr=boondoxo,frak=pxtx,bb=mth]{mathalfa}
\DeclareMathAlphabet{\mathnormal}{T1}{pplx}{m}{it}
\DeclareMathAlphabet{\mathrm}{T1}{pplx}{m}{n}
\DeclareMathAlphabet{\mathit}{T1}{pplx}{m}{it}
\DeclareMathAlphabet{\mathtt}{T1}{lmtt}{m}{n}
\DeclareMathAlphabet{\mathsf}{T1}{kurier}{m}{n}
\DeclareMathAlphabet{\mathbsf}{T1}{kurier}{b}{n}
\DeclareMathAlphabet{\mathbold}{T1}{pplx}{b}{it}
\DeclareMathAlphabet{\mathbf}{T1}{pplx}{b}{n}
\DeclareMathAlphabet{\mathscr}{U}{BOONDOX-calo}{m}{n}
\DeclareMathAlphabet{\mathbscr}{U}{BOONDOX-calo}{b}{n}
%\DeclareMathAlphabet{\mathcal}{OT1}{eusm10}{m}{n}
%\DeclareMathAlphabet{\mathbcal}{OT1}{eusm10}{b}{n}
\DeclareMathAlphabet{\mathfrak}{OT1}{tx-frak}{m}{n}
\DeclareMathAlphabet{\mathbfrak}{OT1}{tx-frak}{b}{n}
\DeclareMathAlphabet{\mathbb}{U}{dsss}{m}{n}
\DeclareSymbolFont{operators}{T1}{uop}{m}{n}

\DeclareSymbolFont{numbers}{T1}{pplx}{m}{n}
\DeclareMathSymbol{0}\mathalpha{numbers}{`0}
\DeclareMathSymbol{1}\mathalpha{numbers}{`1}
\DeclareMathSymbol{2}\mathalpha{numbers}{`2}
\DeclareMathSymbol{3}\mathalpha{numbers}{`3}
\DeclareMathSymbol{4}\mathalpha{numbers}{`4}
\DeclareMathSymbol{5}\mathalpha{numbers}{`5}
\DeclareMathSymbol{6}\mathalpha{numbers}{`6}
\DeclareMathSymbol{7}\mathalpha{numbers}{`7}
\DeclareMathSymbol{8}\mathalpha{numbers}{`8}
\DeclareMathSymbol{9}\mathalpha{numbers}{`9}

%% フォントがない場合は以下の5行を削除
\setsansjfont[Ligatures=TeX,BoldFont=RodinNTLGPro-EB]{FOT-RodinNTLGPro-DB}
\setmainjfont[Ligatures=TeX,BoldFont=RodinNTLGPro-EB]{FOT-RodinNTLGPro-DB}
\setsansfont[Ligatures=TeX,BoldFont=RodinNTLGPro-EB]{FOT-RodinNTLGPro-DB}
\setmainfont[Ligatures=TeX,BoldFont=RodinNTLGPro-EB]{FOT-RodinNTLGPro-DB}
\ltjsetparameter{yjabaselineshift=0pt,yalbaselineshift=0.5pt}

\usepackage{scalefnt}
\usepackage{multirow}
\usepackage{multicol}
\ltjenableadjust[lineend=extended,priority=true,profile=true,linestep=true]
\allowdisplaybreaks[4]

\usepackage{hmtrump}

\ModifyHeading{section}{lines=2}
\ModifyHeading{subsection}{lines=1,format={●#2}}


%%%%%%%%%%%%%%%%%%%%%%%


\begin{document}

\pagestyle{empty}

\vspace{-1cm}
\title{Tapp tarock ルールサマリー}
\author{ひとみさん}

\begin{center}
{\LARGE Tapp Tarock ルール早見}---ひとみさん \today
\end{center}
\bigskip

\setlength{\parindent}{0pt}

\section{ゲームの流れ}
\textbf{\mbox{ディール}\hfill →\hfill\mbox{ビッド}\hfill →\hfill\mbox{交換}\hfill →\hfill
\mbox{宣言}\hfill →\hfill\mbox{トリックテイキング}\hfill →\hfill\mbox{得点計算}\hfill}

ディール: 中央に 3 枚 2 組→全員に 8 枚ずつ 2 回。(16 枚)

\begin{multicols}{2}
	\section{カード(計 54 枚 70 点)}
	\begin{tabular}{r|c}
		\hline
		&強←\hfill→弱\\
		\hline
		切札&\tarottrump{0} \tarottrump{21} \tarottrump{20} \tarottrump{19} …… \tarottrump{2} \tarottrump{1}\\
		\hmC\hmS&\trumpx{K} \trumpx{Q} \trumpx{C} \trumpx{J} \trumpx{T} \trumpx{9} \trumpx{8} \trumpx{7}\\
		\hmH\hmD&\trumpx{K} \trumpx{Q} \trumpx{C} \trumpx{J} \trumpx{1} \trumpx{2} \trumpx{3} \trumpx{4}\\
		\hline
	\end{tabular}\\
	
	\subsection*{点数}
	\begin{tabular}{c|ccccc}
		\hline
		\tarottrump{0} \tarottrump{21} \tarottrump{1}&\trumpx{K}&\trumpx{Q}&\trumpx{C}&\trumpx{J}&他\\
		\hline
		5&5&4&3&2&0\\
		\hline
	\end{tabular}\\
	
	取ったカードを 3 枚セットにして点数の修正:\\
	\begin{tabular}{r|cccc}
		\hline
		点数のあるカード&0&1&2&3\\
		点数のないカード&3&2&1&0\\
		\hline
		点数の修正&+1&±0&-1&-2\\
		\hline
	\end{tabular}\\
	
	デクレアラー目標: 36 点以上獲得
	
	\section{ビッド}
	弱いものから順に、
	\begin{desc}
		\item[ドライアー] 交換あり 基本点: 3 GP
		\item[ウンテラー] 交換あり 基本点: 4 GP
		\item[オーベラー] 交換あり 基本点: 5 GP
		\item[ソロ]       交換なし 基本点: 8 GP
	\end{desc}
	
	ディーラーの右隣から反時計回りに。
	
	先にビッドできる人に優先権。
	
	ソロか最も低いビット(又はホールド)のみ可能。
	
	\section{交換}
	中央 3 枚 2 組を表に→片方を手札に。
	
	3 枚裏向きで捨てる。(デクレアラーが取った扱い)
	
	\begin{itemize}
		\item \tarottrump{0} \tarottrump{21} \tarottrump{1} \trumpx{K} 捨て不可。
		\item ほか切札も可能な限り捨て不可。
	\end{itemize}
	
	取らなかった方はディフェンダーが取った扱い。
	
	\columnbreak
	
	\section{宣言}
	デクレアラーのみ以下の予告が可能:
	\begin{desc}
		\item[バラット] 全トリック取る。
		\item[パガットウルティモ] ラストトリックに\tarottrump{1}で勝つ。
	\end{desc}
	
	お互いダブルをかけることができる:
	\begin{desc}
		\item[コントラ] レコントラ、スブコントラまで可能
		\item[パガットウルティモに対するコントラ] 同上
	\end{desc}
	
	\section{プレイ}
	ディーラーの右隣がオープニングリード。
	
	マストフォロー、できないときは切札を出す。
	
	\section{得点}
	\subsection*{基本点}
	デクレアラーが成功→デクレアラーのみ得点
	
	デクレアラーが失敗→ディフェンダー 2 人が得点
	
	\(\text{得点}=\text{基本点}\times\text{コントラの倍率}\times\text{バラットの倍率}\)
	
	\begin{desc}
		\item[コントラ、レコントラ、スブコントラの倍率]\leavevmode\\
			2, 4, 8 倍
		\item[バラットの倍率]\leavevmode\\
			全トリック取れば 4 倍、宣言すれば更に 2 倍\\
			宣言して失敗は 8 倍
	\end{desc}
	
	\subsection*{パガットウルティモのボーナス}
	ラストトリックに \tarottrump{1} が出たら判定。
	
	デクレアラーが得点 OR ディフェンダー 2 人が得点。
	
	基本 4 GP。ソロ、宣言あり、コントラで倍々になる。
	
	\subsection*{手役}
	個人の点数
	
	\begin{tabular}{rl}
		\hline
		トルル \tarottrump{0}\tarottrump{21}\tarottrump{1} 3 枚&3 GP\\
		トルル \tarottrump{0}\tarottrump{21}\tarottrump{1} 2 枚&1 GP\\
		キング \trumpx{K} 4 枚&3 GP\\
		\hline
	\end{tabular}\\

	
\end{multicols}
\end{document}
