% +++
% latex="lualatex"
% +++
\documentclass[line_length=50zw,head_space=2cm,foot_space=1cm]{jlreq}

\usepackage{luatexja}
\ltjdefcharrange{11}{`→,`↑,`↓,`←}
\ltjsetparameter{jacharrange={-2,-8,+11}}
\usepackage[no-math,match,deluxe,jfm_yoko=jlreq]{luatexja-preset}
\usepackage{luatexja-otf,luatexja-adjust}
\newopentypefeature{PKana}{On}{pkna}
\usepackage{pxrubrica}

\usepackage[osf]{newpxtext}\usepackage{classico}
\usepackage[nowidering]{yhmath}
\usepackage{newpxmath,amsmath,mathtools,mleftright}
\usepackage[T1]{fontenc}
\usepackage[notrig,italicdiff]{physics}
\mleftright

\usepackage[loadonly,]{enumitem}
\newlist{desc}{description}{5}
\setlist[desc]{labelindent=1\zw,labelsep*=1\zw,labelwidth=4\zw,leftmargin=6\zw,}
\newlist{desc5}{description}{5}
\setlist[desc5]{labelindent=1\zw,labelsep*=1\zw,labelwidth=5\zw,leftmargin=6\zw,}
\newlist{enu}{enumerate}{5}
\setlist[enu]{label*=\arabic*.}

\usepackage{Tabbing}

\usepackage[scr=boondoxo,frak=pxtx,bb=mth]{mathalfa}
\DeclareMathAlphabet{\mathnormal}{T1}{pplx}{m}{it}
\DeclareMathAlphabet{\mathrm}{T1}{pplx}{m}{n}
\DeclareMathAlphabet{\mathit}{T1}{pplx}{m}{it}
\DeclareMathAlphabet{\mathtt}{T1}{lmtt}{m}{n}
\DeclareMathAlphabet{\mathsf}{T1}{kurier}{m}{n}
\DeclareMathAlphabet{\mathbsf}{T1}{kurier}{b}{n}
\DeclareMathAlphabet{\mathbold}{T1}{pplx}{b}{it}
\DeclareMathAlphabet{\mathbf}{T1}{pplx}{b}{n}
\DeclareMathAlphabet{\mathscr}{U}{BOONDOX-calo}{m}{n}
\DeclareMathAlphabet{\mathbscr}{U}{BOONDOX-calo}{b}{n}
%\DeclareMathAlphabet{\mathcal}{OT1}{eusm10}{m}{n}
%\DeclareMathAlphabet{\mathbcal}{OT1}{eusm10}{b}{n}
\DeclareMathAlphabet{\mathfrak}{OT1}{tx-frak}{m}{n}
\DeclareMathAlphabet{\mathbfrak}{OT1}{tx-frak}{b}{n}
\DeclareMathAlphabet{\mathbb}{U}{dsss}{m}{n}
\DeclareSymbolFont{operators}{T1}{uop}{m}{n}

\DeclareSymbolFont{numbers}{T1}{pplx}{m}{n}
\DeclareMathSymbol{0}\mathalpha{numbers}{`0}
\DeclareMathSymbol{1}\mathalpha{numbers}{`1}
\DeclareMathSymbol{2}\mathalpha{numbers}{`2}
\DeclareMathSymbol{3}\mathalpha{numbers}{`3}
\DeclareMathSymbol{4}\mathalpha{numbers}{`4}
\DeclareMathSymbol{5}\mathalpha{numbers}{`5}
\DeclareMathSymbol{6}\mathalpha{numbers}{`6}
\DeclareMathSymbol{7}\mathalpha{numbers}{`7}
\DeclareMathSymbol{8}\mathalpha{numbers}{`8}
\DeclareMathSymbol{9}\mathalpha{numbers}{`9}

%% フォントがない場合は以下の5行を削除
%\setmainfont[
%	Ligatures=TeX,
%	Scale=0.98,
%]{Palatino}
%\setsansfont[
%	Ligatures=TeX,
%	Scale=0.98,
%]{Palatino}
\setmainjfont[
	Ligatures=TeX,
	JFM=jlreq,
	BoldFont=FOT-TsukuAOldMinPr6-E,
]{FOT-TsukuAOldMinPr6-R}
\setsansjfont[
	Ligatures=TeX,
	JFM=jlreq,
	BoldFont=TsukuOldGothicStd-B,
	%AutoFakeBold=1.5,
]{FOT-TsukuOldGothicStd-B}
\setmonofont[
	Ligatures=TeXReset,
]{HackGen}
\setmonojfont[
	Ligatures=TeXReset,
]{HackGen}

%\ltjsetparameter{yjabaselineshift=0pt,yalbaselineshift=0.5pt}

\usepackage{scalefnt}
\usepackage{multirow}
\usepackage{multicol}
\ltjenableadjust[lineend=extended,priority=true,profile=true,linestep=true]
\allowdisplaybreaks[4]

\usepackage{xiangqi}
\newcommand{\特殊}{\textsuperscript ☆}

\ModifyHeading{section}{lines=2}
\ModifyHeading{subsection}{lines=1}

%%%%%%%%%%%%%%%%%%%%%%%


\begin{document}

\pagestyle{empty}

\vspace{-1cm}
\title{シャンチー麻雀}
\author{ひとみさん}

\begin{center}
	{\LARGE \jruby[g]{象棋}{シャンチー}麻雀}(32 牌、\textcolor{lightgray}{2,} 3, \textcolor{lightgray}{4} 人)%
	---ひとみさん \today
\end{center}

\setlength{\parindent}{0pt}

\section{牌構成}
%\vspace{-3\zh}
\begin{table}[h]
	\centering\Huge
	\newlength{\TableWidth}
	\settowidth{\TableWidth}{\xq{R}{K}\xq{R}{A}\xq{R}{E}\ \xq{R}{H}\xq{R}{R}\xq{R}{C}\ \xq{R}{P}\xq{R}{P}\xq{R}{P}}
	\begin{minipage}{\TableWidth}
		\setlength{\baselineskip}{0\zh}
		\begin{Tabbing}
			\xq{R}{K}\TAB=\xq{R}{A}\xq{R}{E}\ \TAB=\xq{R}{R}\xq{R}{H}\xq{R}{C}\ \TAB=\hspace{.65\zw}\TAB=\kill
			\TAB>\makebox[0pt][r]{\xq{R}{K}}\xq{R}{A}\xq{R}{E}\TAB>\xq{R}{R}\xq{R}{H}\xq{R}{C}\TAB>
				\xq{R}{P}\xq{R}{P}\xq{R}{P}\\
			\TAB>\xq{R}{A}\xq{R}{E}\TAB>\xq{R}{R}\xq{R}{H}\xq{R}{C}\TAB>\TAB>\xq{R}{P}\xq{R}{P}\\
			\TAB>\makebox[0pt][r]{\xq{B}{K}}\xq{B}{A}\xq{B}{E}\TAB>\xq{B}{H}\xq{B}{R}\xq{B}{C}\TAB>
				\xq{B}{P}\xq{B}{P}\xq{B}{P}\\
			\TAB>\xq{B}{A}\xq{B}{E}\TAB>\xq{B}{R}\xq{B}{H}\xq{B}{C}\TAB>\TAB>\xq{B}{P}\xq{B}{P}
		\end{Tabbing}
	\end{minipage}
\end{table}
%\vspace{-3\zh}

\section{和了形}
%\vspace{-1\zh}
手牌は 4 枚。\textbf{1 面子 1 雀頭}で和了。1 点縛り(懸賞役は含まない)。王牌無し。

雀頭は同一牌 2 枚。

面子は順子 \xq{R}{K}\xq{R}{A}\xq{R}{E}/\xq{R}{R}\xq{R}{H}\xq{R}{C}/%
\xq{B}{K}\xq{B}{A}\xq{B}{E}/\xq{B}{R}\xq{B}{H}\xq{B}{C}(吃可)
または刻子 \xq{R}{P}\xq{R}{P}\xq{R}{P}/\xq{B}{P}\xq{B}{P}\xq{B}{P}(碰可) のみ。

吃は上家から、碰は全員からできる。二人が碰したときは上家優先。槓子は無し。

\section{役一覧}
%\vspace{-3\zh}
\begin{multicols}{2}
	\subsection*{6 点役(役満)}
	\begin{desc5}
		\item[天和] 親の配牌での和了。
		\item[地和] 子の第一自摸での和了。
		\item[三公\特殊]
			{\Large \xq{1}{K}\xq{1}{A}\xq{1}{A}\xq{1}{E}\xq{1}{E}}
		\item[五兵卒]
			{\Large \xq{1}{P}\xq{1}{P}\xq{1}{P}\xq{1}{P}\xq{1}{P}}
	\end{desc5}
	\subsection*{3 点役}
	\begin{desc5}
		\item[将帥同国\特殊]
			{\Large \xq{R}{K}\xq{B}{K}\xq{1}{R}\xq{1}{H}\xq{1}{C}}
	\end{desc5}
	\subsection*{2 点役}
	\begin{desc5}
		\item[将帥領兵\特殊]
			{\Large \xq{R}{K}\xq{B}{K}\xq{1}{P}\xq{1}{P}\xq{1}{P}}
		\item[兵車行]
			{\Large \xq{1}{R}\xq{1}{R}\xq{1}{P}\xq{1}{P}\xq{1}{P}}
	\end{desc5}
	\columnbreak
	\subsection*{1 点役}
	\begin{desc}
		\item[無兵卒] \xq{R}{P}\xq{B}{P} の無い和了。
		\item[清一色] 一色のみで和了。
		\item[門前自摸和] 門前での自摸和。
		\item[海底摸月] 最後の自摸牌での自摸和。
		\item[河底撈魚] 最後の打牌での栄和。
		\item[立直] 供託 1 点。
		\item[一発] 立直後一順以内で和了。
	\end{desc}
	\subsection*{その他}
	\begin{desc}
		\item[馬到成功] \textbf{1 点}。\xq{R}{H} または \xq{B}{H} で和了。懸賞役。
		\item[一炮而紅] \textbf{1 点}。\xq{R}{C} または \xq{B}{C} で和了。懸賞役。
		\item[積み棒] 親が連荘する度に \textbf{1 点加算}。
		\item[親] 和了時、親の支払・受取は \textbf{1 点加算}。
		\item[不聴罰符] 流局時、不聴ならば \textbf{1 点供託}。
	\end{desc}
\end{multicols}
%\vspace{-1\zh}
※無兵卒、清一色は 2, 3 点役に複合せず。
※青色の牌は同色の牌を示す。
※☆は特殊形の和了。

%\vspace{-1\zh}
\section{細則}
%\vspace{-1\zh}
\begin{itemize}
	\item 役の複合に依る点数の加算は、親・積み棒の加点は除いて、6 点迄。
		ダブル役満を認めるかは事前に取決め。
	\item 振聴は現物のみ栄和不可。
	\item 自摸和の点数は、全員満額払か、分割して支払、どちらか事前に取決め。
		分割する場合、端数は切上げ。
	\item 錯和は全員に 1 点払い。(3 点でも可)
	\item その他、リーチ麻雀に準じる。
\end{itemize}
\end{document}

