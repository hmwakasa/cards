% +++
% latex="lualatex"
% +++
\documentclass[line_length=50zw,head_space=2cm,foot_space=1cm]{jlreq}

\usepackage{luatexja}
\ltjdefcharrange{11}{`→,`↑,`↓,`←}
\ltjsetparameter{jacharrange={-2,-8,+11}}
\usepackage[no-math,match,deluxe,jfm_yoko=jlreq]{luatexja-preset}
\usepackage{luatexja-otf,luatexja-adjust}
\newopentypefeature{PKana}{On}{pkna}

\usepackage[nowidering]{yhmath}
\usepackage{newpxmath,amsmath,mathtools,mleftright}
\usepackage[T1]{fontenc}
\usepackage[notrig,italicdiff]{physics}
\mleftright

\usepackage[loadonly,]{enumitem}
\newlist{desc}{description}{5}
\setlist[desc]{labelindent=1\zw,labelsep*=1\zw,labelwidth=4\zw}
\newlist{enu}{enumerate}{5}
\setlist[enu]{label*=\arabic*.}

\usepackage[scr=boondoxo,frak=pxtx,bb=mth]{mathalfa}
\DeclareMathAlphabet{\mathnormal}{T1}{pplx}{m}{it}
\DeclareMathAlphabet{\mathrm}{T1}{pplx}{m}{n}
\DeclareMathAlphabet{\mathit}{T1}{pplx}{m}{it}
\DeclareMathAlphabet{\mathtt}{T1}{lmtt}{m}{n}
\DeclareMathAlphabet{\mathsf}{T1}{kurier}{m}{n}
\DeclareMathAlphabet{\mathbsf}{T1}{kurier}{b}{n}
\DeclareMathAlphabet{\mathbold}{T1}{pplx}{b}{it}
\DeclareMathAlphabet{\mathbf}{T1}{pplx}{b}{n}
\DeclareMathAlphabet{\mathscr}{U}{BOONDOX-calo}{m}{n}
\DeclareMathAlphabet{\mathbscr}{U}{BOONDOX-calo}{b}{n}
%\DeclareMathAlphabet{\mathcal}{OT1}{eusm10}{m}{n}
%\DeclareMathAlphabet{\mathbcal}{OT1}{eusm10}{b}{n}
\DeclareMathAlphabet{\mathfrak}{OT1}{tx-frak}{m}{n}
\DeclareMathAlphabet{\mathbfrak}{OT1}{tx-frak}{b}{n}
\DeclareMathAlphabet{\mathbb}{U}{dsss}{m}{n}
\DeclareSymbolFont{operators}{T1}{uop}{m}{n}

\DeclareSymbolFont{numbers}{T1}{pplx}{m}{n}
\DeclareMathSymbol{0}\mathalpha{numbers}{`0}
\DeclareMathSymbol{1}\mathalpha{numbers}{`1}
\DeclareMathSymbol{2}\mathalpha{numbers}{`2}
\DeclareMathSymbol{3}\mathalpha{numbers}{`3}
\DeclareMathSymbol{4}\mathalpha{numbers}{`4}
\DeclareMathSymbol{5}\mathalpha{numbers}{`5}
\DeclareMathSymbol{6}\mathalpha{numbers}{`6}
\DeclareMathSymbol{7}\mathalpha{numbers}{`7}
\DeclareMathSymbol{8}\mathalpha{numbers}{`8}
\DeclareMathSymbol{9}\mathalpha{numbers}{`9}

%% フォントがない場合は以下の5行を削除
\setsansjfont[Ligatures=TeX,BoldFont=RodinNTLGPro-EB]{FOT-RodinNTLGPro-DB}
\setmainjfont[Ligatures=TeX,BoldFont=RodinNTLGPro-EB]{FOT-RodinNTLGPro-DB}
\setsansfont[Ligatures=TeX,BoldFont=RodinNTLGPro-EB]{FOT-RodinNTLGPro-DB}
\setmainfont[Ligatures=TeX,BoldFont=RodinNTLGPro-EB]{FOT-RodinNTLGPro-DB}
\ltjsetparameter{yjabaselineshift=0pt,yalbaselineshift=0.5pt}

\usepackage{scalefnt}
\usepackage{multirow}
\usepackage{multicol}
\ltjenableadjust[lineend=extended,priority=true,profile=true,linestep=true]
\allowdisplaybreaks[4]

\usepackage{hmtrump}

\newcommand{\hmG}{%
	\tikz[baseline=(T.base)]
	\node[fill=black,text=white,outer sep=0pt,inner sep=0.1ex]
	(T)at(0,0){G};%
}

%%%%%%%%%%%%%%%%%%%%%%%


\begin{document}

\pagestyle{empty}

\vspace{-1cm}
\title{fipsenルールサマリー}
\author{ひとみさん}

\begin{center}
{\LARGE Fipsen ルール早見}---ひとみさん \today
\end{center}

\setlength{\parindent}{0pt}

\section{ゲームの流れ}
\textbf{\mbox{カードを配る}\hfill →\hfill \mbox{ビッドをする}\hfill →\hfill
\mbox{スカート交換、切り札決め}\hfill →\hfill \mbox{トリックテイキング}\hfill
→\hfill \mbox{得点計算}\hfill}\\
\textbf{配り方}: 3→スカート2→2 (配り残り: 3{\small [4 人]} or 8{\small [3 人]})\\
ディーラーの次がオープニングリード\hspace*{1\zh}ビッドは勝ち抜き方式(先のビッドに優先権)

\begin{multicols}{2}
	\section{カード}
	\begin{tabular}{c|r}
		\hline
		&強←\hfill→弱\\
		\hline
		\hmC\hmS\hmH&\trumpx{A} \trumpx{K} \trumpx{Q} \trumpx{J} \trumpx{T} \trumpx{9} \trumpx{8} \trumpx{7}\\
		\hline
		\hmD&\trumpx{7}\\
		\hline
	\end{tabular}
	
	25 枚
	
	\section{ビッド}
	取るトリック数と追加宣言をビッド
	\begin{desc}
		\item[ハント] 手札交換無し
		\item[ルーテン] \hmD 切札
		\item[ドゥルヒ] 5 トリック取る
	\end{desc}
	
	同じ数のビッドなら追加宣言が多いほうが強い
	
	先のビッドに優先権の勝ち抜き方式
	
	5 トリックのビッドには自動的にドゥルヒ追加
	
	5 未満ビッドでもドゥルヒ宣言だと 5 トリック必要
	
	\subsection*{特殊なビッド}
	\begin{desc}
		\item[キーカー] 手札に \trumpx{K} \trumpx{Q} \trumpx{J} がない時に可能\\
			配り残りからも交換可\hspace{1\zw}全トリック取る\\
			4 トリックと 5 トリックの間の強さ
	\end{desc}
	\columnbreak
	
	\section{プレイ}
	\textbf{追加宣言は後付可能}
	
	デクレアラーがスカートを手札に加えて 2 枚捨札\\
	(やらなければハント追加)
	
	デクレアラーが切り札を宣言してプレイ\\
	(\hmD 切札ならルーテン追加)
	
	ビッド数のトリックを取ったらそこで終了\\
	(ビッド数を超えてプレイしたらドゥルヒ追加)
	
	\subsection*{キーカー時}
	デクレアラーは左隣に手札に絵札がないことを見せる
	
	スカートと配り残りを手札に加えて 5 枚捨てる\\
	{\small (3 人時: 手札を捨て配残とスカートから 5 枚手札)}
	
	切札宣言(ルーテン追加可能)
	
	全トリック取れなさそうならば降参可能
	
	\section{得点}
	\(\text{基本点}=\text{ビッド数}\times2^\text{追加宣言数}\)
	
	成功なら基本点を得点
	
	失敗なら基本点の 2 倍失点
	
	キーカーの基本点は 10\\
	(ルーテンつけたならば 2 倍)
	
	キーカーの降参は 5 点失点
	
	\subsection*{ズィーベナーフィップス}
	手札が \trumpx{7}\trumpx{7}\trumpx{7}\trumpx{7}\trumpx{A} なら宣言\\
	(キーカー含む手札交換後でも宣言可)
	
	30 点得点、そのディール終了
\end{multicols}
\end{document}
